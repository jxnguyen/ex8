\documentclass[a4paper, 12pt]{article}

\usepackage[ngerman]{babel}                 % german
\usepackage[margin=3cm]{geometry}           % margin
\setlength\parindent{0pt}                   % no indentation
\usepackage{enumerate}                      % lists
\usepackage{verbatim}												% verbatim input from text file
\usepackage{listings}                       % code listings
\usepackage{color}

\definecolor{gray}{rgb}{0.5,0.5,0.5}
\definecolor{lightgray}{rgb}{0.7,0.7,0.7}
\definecolor{orange}{rgb}{0.87, 0.56, 0.4}

\lstset{
	frame=tb,
	framesep = 1cm,
  language=Python,
  aboveskip=3mm,
  belowskip=3mm,
  showstringspaces=false,
  columns=flexible,
  basicstyle={\small\ttfamily},
  numbers=left,
  numberstyle=\tiny\color{gray},
  keywordstyle=\color{red},
  commentstyle=\color{lightgray},
  stringstyle=\color{orange},
  breaklines=true,
  breakatwhitespace=true,
  tabsize=3
}


% COMMANDS
% --------------------------------------------------------
\newcommand{\TITLE}[3]{\Large\textbf{#1:} \"Ubung #2 \hfill #3\\[3pt]}
\newcommand{\TUTOR}[3]{\normalsize{#1 (#2 #3 Uhr)}}
\newcommand{\AUTHOR}{\normalsize{John Nguyen \& Anastasia Isaeva}}
\newcommand{\AUFGABE}[3]{
\vspace{1.5cm}
\noindent\large\textbf{Aufgabe #1: }{#2}\hfill\small#3 Punkte\\[-5pt]
\rule{\textwidth}{0.4pt}\\}


% DOCUMENT
% --------------------------------------------------------
\begin{document}

\noindent\TITLE{OOP}{8}{10.06.16}
\TUTOR{Christoph Van Heteren-Frese}{Mi}{8}\\
\AUTHOR

% ---------------------------------------------------------------------------
\AUFGABE{38}{Stabile Sortierverfahren}{10}

\begin{description}
	\item[Bubble Sort]
	ist ein stabiler Sortierverfahren, aber man muss bei der Implementierung auf die Vergleichsoperation (Zeile 10) achten. Bubblesort ist stabil, wenn der Vergleich von zwei benachbarten Elementen \"uberpr"uft, ob das erste Element \emph{echt gr"o"ser} als das zweite Element. Wenn die Operation $\ge$ ist, dann ist Bubblesort nicht mehr stabil, denn die Positionen von zwei gleichen Elementen werden dann getauscht.

	\begin{lstlisting}
		def bubblesort(A):
			# end index of unsorted part
			end = len(A) - 1
			# while still unsorted
			while end > 0:
				last_change = 0
				# for each elem in unsorted
				for i in range(end):
					# if out of order
					if A[i] > A[i+1]:
						# swap
						A[i], A[i+1] = A[i+1], A[i]
						# note index
						last_change = i
				# update new end index
				end = last_change
	\end{lstlisting}

	\item[Merge Sort]
	ist ein stabiler Sortierverfahren, wenn man wie beim Bubblesort die Vergleichsoperation (Zeile 6) ber"ucksichtigt. An dieser Stelle wird \"uberpr"uft, welches Element von zwei Listen in die neue verschmolzene Liste eingef"ugt werden soll. Im Bez"ug auf die urspr"ungliche Reihenfolge kommen die Elemente in $X$ vor den Elementen in $Y$ vor. Das hei"st, mit der $\le$ Operation wird diese Reihenfolge beim gleichen Elementen belassen. Falls diese Operation $>$ ist, dann ist der Algorithmus nicht stabil.

	\begin{lstlisting}
		def mergesort(A):
			def merge(X, Y):
				result = []
				# while elements in both lists
				while X and Y:
					result.append(X.pop(0) if X[0] <= Y[0] else Y.pop(0))
				# consume remaining elems
				return result + (X if X else Y)

			# base case
			if len(A) <= 1: return A

			# split lists & merge
			mid   = len(A)//2
			left  = mergesort(A[:mid])
			right = mergesort(A[mid:])
			return merge(left, right)
	\end{lstlisting}

	\item[Insertion Sort]
	ist auch stabil, wenn die passende Operation verwendet wird. In der Schleife an Zeile 9 wird das aktuelles Element $e$ in die sortierte Teilliste eingef"ugt. An Zeile 11 wird die richtige Position innerhabl dieser Liste berechnet. Falls ein Element $x$, das gleich $e$ ist, schon in der sortierten Liste liegt, 

	\begin{lstlisting}
		def insertionsort(A):
			# sorted list
			result = []
			# while elements
			while A:
				# first elem
				e = A.pop(0)
				# for x in sorted
				for x in result:
					# if elem smaller
					if e < x:
						# insert before x
						result.insert(result.index(x), e)
						break
				else:
					# add to end
					result.append(e)

			return result
	\end{lstlisting}
\end{description}

\end{document}
